% template.tex, dated April 5 2013
% This is a template file for Annual Reviews 1 column Journals
%
% Compilation using ar-1col-S2O.cls' - version 1.0, Aptara Inc.
% (c) 2013 AR
%
% Steps to compile: latex latex latex
%
% For tracking purposes => this is v1.0 - Apr. 2013

\documentclass{ar-1col-S2O}
\usepackage[numbers]{natbib}
\usepackage{url}
\setcounter{secnumdepth}{4}

% Metadata Information
\jname{Xxxx. Xxx. Xxx. Xxx.}
\jvol{AA}
\jyear{YYYY}
\doi{10.1146/((please add article doi))}


% Document starts
\begin{document}

% Page header
\markboth{Author et al.}{Short title}

% Title
\title{Machine Learning in Biology}


%Authors, affiliations address.
\author{Author B. Authorone,$^1$ Firstname C. Authortwo,$^2$ and D. Name Authorthree$^3$
\affil{$^1$Department/Institute, University, City, Country, Postal code; email: author@email.edu}
\affil{$^2$Department/Institute, University, City, Country, Postal code}
\affil{$^3$Department/Institute, University, City, Country, Postal code}}

%Abstract
\begin{abstract}
Abstract text, approximately 150 words. 
\end{abstract}

%Keywords, etc.
\begin{keywords}
keywords, separated by comma, no full stop, lowercase
\end{keywords}
\maketitle

%Table of Contents
\tableofcontents


% Heading 1
\section{INTRODUCTION}
Please begin the main text of your article here. 



%Heading 1
\section{FIRST-LEVEL HEADING}
This is dummy text. 
% Heading 2
\subsection{Second-Level Heading}
This is dummy text. This is dummy text. This is dummy text. This is dummy text.

% Heading 3
\subsubsection{Third-Level Heading}
This is dummy text. This is dummy text. This is dummy text. This is dummy text. 

% Heading 4
\paragraph{Fourth-Level Heading} Fourth-level headings are placed as part of the paragraph.

%Example of a Figure
\section{ELEMENTS\ OF\ THE\ MANUSCRIPT} 
\subsection{Figures}Figures should be cited in the main text in chronological order. This is dummy text with a citation to the first figure (\textbf{Figure \ref{fig1}}). Citations to \textbf{Figure \ref{fig1}} (and other figures) will be bold. 

\begin{figure}[h]
\includegraphics[width=3in]{SampleFigure}
\caption{Figure caption with descriptions of parts a and b}
\label{fig1}
\end{figure}

% Example of a Table
\subsection{Tables} Tables should also be cited in the main text in chronological order (\textbf {Table \ref{tab1}}).

\begin{table}[h]
\tabcolsep7.5pt
\caption{Table caption}
\label{tab1}
\begin{center}
\begin{tabular}{@{}l|c|c|c|c@{}}
\hline
Head 1 &&&&Head 5\\
{(}units)$^{\rm a}$ &Head 2 &Head 3 &Head 4 &{(}units)\\
\hline
Column 1 &Column 2 &Column3$^{\rm b}$ &Column4 &Column\\
Column 1 &Column 2 &Column3 &Column4 &Column\\
Column 1 &Column 2 &Column3 &Column4 &Column\\
Column 1 &Column 2 &Column3 &Column4 &Column\\
\hline
\end{tabular}
\end{center}
\begin{tabnote}
$^{\rm a}$Table footnote; $^{\rm b}$second table footnote.
\end{tabnote}
\end{table}

% Example of lists
\subsection{Lists and Extracts} Here is an example of a numbered list:
\begin{enumerate}
\item List entry number 1,
\item List entry number 2,
\item List entry number 3,\item List entry number 4, and
\item List entry number 5.
\end{enumerate}

Here is an example of a extract.
\begin{extract}
This is an example text of quote or extract.
This is an example text of quote or extract.
\end{extract}

\subsection{Sidebars and Margin Notes}
% Margin Note
\begin{marginnote}[]
\entry{Term A}{definition}
\entry{Term B}{definition}
\entry{Term C}{defintion}
\end{marginnote}

\begin{textbox}[h]\section{SIDEBARS}
Sidebar text goes here.
\subsection{Sidebar Second-Level Heading}
More text goes here.\subsubsection{Sidebar third-level heading}
Text goes here.\end{textbox}



\subsection{Equations}
% Example of a single-line equation
\begin{equation}
a = b \ {\rm ((Single\ Equation\ Numbered))}
\end{equation}
%Example of multiple-line equation
Equations can also be multiple lines as shown in Equations 2 and 3.
\begin{eqnarray}
c = 0 \ {\rm ((Multiple\  Lines, \ Numbered))}\\
ac = 0 \ {\rm ((Multiple \ Lines, \ Numbered))}
\end{eqnarray}

% Summary Points
\begin{summary}[SUMMARY POINTS]
\begin{enumerate}
\item Summary point 1. These should be full sentences.
\item Summary point 2. These should be full sentences.
\item Summary point 3. These should be full sentences.
\item Summary point 4. These should be full sentences.
\end{enumerate}
\end{summary}

% Future Issues
\begin{issues}[FUTURE ISSUES]
\begin{enumerate}
\item Future issue 1. These should be full sentences.
\item Future issue 2. These should be full sentences.
\item Future issue 3. These should be full sentences.
\item Future issue 4. These should be full sentences.
\end{enumerate}
\end{issues}

%Disclosure
\section*{DISCLOSURE STATEMENT}
If the authors have noting to disclose, the following statement will be used: The authors are not aware of any affiliations, memberships, funding, or financial holdings that
might be perceived as affecting the objectivity of this review. 

% Acknowledgements
\section*{ACKNOWLEDGMENTS}
Acknowledgements, general annotations, funding.

% References
%
% Margin notes within bibliography
\section*{LITERATURE\ CITED}

To download the appropriate bibliography style file, please see \url{https://www.annualreviews.org/page/authors/general-information}. 

\\

\noindent
Please see the Style Guide document for instructions on preparing your Literature Cited.

The citations should be listed in order of appearance, with titles. For example:






\begin{verbatim}
\begin{thebibliography}{00}
\bibitem{Trouve1995a}
Trouv\'{e} A. 1995. {\it An approach of pattern recognition through infinite 
dimensional group action.} Rep. LMENS-95-9, Lab. Math. l'Ecole Norm. Superieure, Paris 

\bibitem{Christensen1996}
Christensen G, Miller MI, Rabbit RD. 1995. Deformable templates
using large deformation kinematics. {\it IEEE Trans. Med. Imaging}
5(10):1435--47

\bibitem{Grenander1998}
Grenander U, Miller MI. 1998. Computational anatomy: an emerging
discipline. {\it Q. Appl. Math.} 56:617--94

\bibitem{Dupuis1998}
Dupuis P, Grenander U, Miller MI. 1998. Variation problems on
flows of diffeomorphisms for image matching. {\it Q. Appl. Math.}
56:587--600

\bibitem{Miller-Younes-2001}
Miller MI, Younes L. 2001. Group actions, homeomorphisms, and matching: a general framework. {\it Int. J. Comput. Vis.}
41:61--84

\bibitem{Toga2001}
Toga A, Thompson PM. 2001. Maps of the brain. {\it Anat. Rec.}
265:37--53



\end{thebibliography}
\end{verbatim}




\begin{thebibliography}{00}

\bibitem{Trouve1995a}%1
Trouv\'{e} A. 1995. {\it An approach of pattern recognition through
infinite dimensional group action.} Rep. LMENS-95-9, Lab. Math. l'Ecole Norm. Superieure, Paris  

\bibitem{Christensen1996}%2
Christensen G, Miller MI, Rabbit RD. 1995. Deformable templates
using large deformation kinematics. {\it IEEE Trans. Med. Imaging}
5(10):1435--47

\bibitem{Grenander1998}%3
Grenander U, Miller MI. 1998. Computational anatomy: an emerging
discipline. {\it Q. Appl. Math.} 56:617--94

\bibitem{Dupuis1998}%4
Dupuis P, Grenander U, Miller MI. 1998. Variation problems on
flows of diffeomorphisms for image matching. {\it Q. Appl. Math.}
56:587--600

\bibitem{Miller-Younes-2001}%5
Miller MI, Younes L. 2001. Group actions, homeomorphisms, and matching: a general framework. {\it Int. J. Comput. Vis.}
41:61--84

\bibitem{Toga2001}%6
Toga A, Thompson PM. 2001. Maps of the brain. {\it Anat. Rec.}
265:37--53



 

\end{thebibliography}


\end{document}
